\section{Đường cong elliptic trên trường hữu hạn}

Trong phần trước, chúng ta đã phát triển lý thuyết về đường cong elip về mặt hình học.
Tuy nhiên, để áp dụng lý thuyết về đường cong elliptic vào mật mã,
chúng ta cần xem xét các đường cong elliptic mà các điểm của nó có tọa độ trong một trường hữu hạn $F_p$.
Các ứng dụng về mật mã của đường cong Elliptic đa số chỉ sử dụng các đường
cong trên trường hữu hạn.

\subsection{Sơ bộ}
\begin{definition}
	Trường là một tập hợp K có nhiều hơn một phần tử, được định nghĩa hai phép toán cộng và nhân,
	ký hiệu bởi dấu $(+)$ và dấu $(.)$. Trường thỏa mãn các tính chất của số học.
\end{definition}
Các tính chất số học:
TODO:
\begin{enumerate}
	\item Tính kết hợp
	\item Tính giao hoán
	\item Đơn vị cộng và đơn vị nhân
	\item Nghịch đảo phép cộng
	\item Nghịch đảo phép nhân
	\item Tính phân phối
\end{enumerate}

\begin{definition}
	Trường hữu hạn (còn gọi là trường Galois) là những trường có hữu hạn số phần tử.
	Bậc của một trường hữu hạn là số phần tử của nó, là số nguyên tố hoặc lũy thừa nguyên tố.
\end{definition}
Trường hữu hạn là cơ bản trong một số lĩnh vực toán học và khoa học máy tính,
bao gồm lý thuyết số, hình học đại số, lý thuyết Galois, hình học hữu hạn, mật mã và lý thuyết mã hóa.

\begin{definition}[Bình phương modulo]
	Cho số nguyên dương $m \geq 2$. Số nguyên $a$ được gọi là \textit{bình phương modulo $m$} nếu $\gcd(a,m) = 1$ và phương trình
	$$x^2 \equiv a (\mod{m})$$
	có nghiệm
\end{definition}

\begin{definition}[Nghịch đảo modulo]
	Với một số nguyên $a$, ta gọi nghịch đảo modulo $m$ của $a$ là $a^{-1}$ là số nguyên thỏa mãn:
	$$a * a^{-1} \equiv 1 (\mod{m})$$
	Chú ý rằng không phải lúc nào $a^{-1}$ cũng tồn tại. Ví dụ với $m = 4, a = 2$, ta không thể tìm được $a^{-1}$ thỏa mãn đằng thức trên.
\end{definition}

\begin{definition}[Thặng dư bình phương]
	Một số nguyên $q$ gọi là thặng dư bình phương theo modulo $m$ nếu nó đồng dư với một số chính phương theo modulo $m$.
	Nói cách khác, tồn tại số nguyên $x$ thỏa mãn:
	$$x^2 \equiv q (\mod{m})$$
	Ngược lại, $q$ được gọi là \textit{phi thặng dư bình phương}

\end{definition}

\begin{definition}[Modular square root]
	Một Modular square root $r$ của số nguyên $a$ theo modulo $m$ là một số nguyên thỏa mãn:
	$$r^2 \equiv a (\mod{m})$$
\end{definition}

Xét $F_p$ là một trường hữu hạn (hữu hạn số phần tử nguyên dương):
$$F_p = \{0, 1, 2, \ldots p-1\}$$
Với $p$ là một số nguyên tố. $F_p$ giống như cách viết $Z/mZ$ là vành các số nguyên modulo $m$.
\subsection{Đường cong elliptic trên trường hữu hạn}

Ta định nghĩa một đường cong elliptic $E$ trên trường hữu hạn $\mathbb{F}_q$ là một phương trình có dạng:
$$E: Y^2 = X^3 + AX + B\ \text{với các hằng số}\ A, B \in F_p\ \text{thỏa mãn}\ 4A^3 + 27B^2 \neq 0$$


Tập hợp các điểm trên $E$ có toạ độ thuộc $\mathbb{F} _p$ được kí hiệu bởi
$$E(F_p) = \{(x, y) : x, y \in \mathbb{F}_p\ \text{thỏa mãn}\ y^2 = x^3 + A x + B\} \cup \mathcal{O}$$

\begin{remark}
	Vì một vài lí do mà chúng tôi sẽ giải thích ở phần sau, ở đây, chúng tôi thêm điều kiện $p \geq 3$.
	Những đường cong Elliptic trên trường $\mathbb{F}_2$ có vai trò quan trọng trong mật mã, nhưng chúng rất phức tạp, nên chúng ta sẽ thảo luận về chúng ở phần 7.
\end{remark}

\begin{example}
	\label{ex:ex4}
	Xem xét đường cong elliptic
	$$E: Y^2 = X^3 + 3X + 8 \ \ \ \ \text{trên trường}\  F_{13}\ $$
\end{example}

Ta tìm các điểm thuộc $E(\mathbb{F}_{13})$ bằng cách thay tất cả giá trị của $X = 0,1,2,\ldots,12$ và kiểm tra với mỗi giá trị của $X$, $X^3 + 3X + 8$ có là bình phương modulo của 13 hay không.
Ví dụ, thay $X = 0$, ta có $X^3+3X+8 = 8$ và 8 không phải bình phương modulo của 13.
Tiếp theo, thay $X = 1$, ta được $X^3+3X+8 = 12$ và 12 là bình phương modulo của 13. Nó có 2 nghiệm
$$5^2 \equiv 12 (\mod{13}) \ \ \ \ \text{và} \ \ \ \ 8^2 \equiv 12 (\mod 13)$$.

Ta tìm được 2 điểm $(1,5)$ và $(1,8)$ thuộc $E(\mathbb{F}_{13})$. Tiếp tục theo cách này, ta kết thúc với tập hoàn chỉnh gồm 9 điểm:
$$E(\mathbb{F}_{13}) = \{ \mathcal{O}, (1,5), (1,8), (2,3), (2,10), (9,6), (9,7), (12,2), (12,11)\}.$$

Cho $P_1 = (x_1, y_1)$ và $P_2 = (x_2, y_2)$ thuộc $E(\mathbb{F}_{p})$. Ta định nghĩa tổng $P_1+P_2$ có tọa độ $(x_3, y_3)$ thu được bằng cách áp dụng thuật toán cộng (\ref{th:th2}).
Vì tọa độ các độ các điểm đó nằm trong trường $\mathbb{F}_p$, ta thu được $(x_3, y_3)$ có tọa độ trong trường $\mathbb{F}_p$. Nhưng điều này vẫn chưa đủ chỉ ra
$(x_3, y_3)$ có thể thuộc $E(\mathbb{F}_{p})$ hay không.

\begin{theorem}
	\label{th:th3}
	Cho $E$ là đường cong elliptic trên $\mathbb{F}_p$ và $P$ và $Q$ là 2 điểm thuộc $E(\mathbb{F}_{p})$.
	\begin{itemize}
		\item Thuật toán cộng đường cong elliptic áp dụng cho $P$ và $Q$ (\ref{th:th2}) đưa ra một điểm trong $E(\mathbb{F}_{p})$.
		      Điểm này được kí hiệu bởi $P + Q$.
		\item Luật cộng trên $E(\mathbb{F}_{p})$ thỏa mãn tất cả các tính chất được liệt kê ở định lý \ref{th:th1}.
		      Nói cách khác, luật cộng này làm cho $E(\mathbb{F}_{p})$ thành nhóm hữu hạn.
	\end{itemize}
\end{theorem}

\begin{proof}

\end{proof}

\begin{example}
	\label{ex:ex5}
	Tiếp tục với đường cong $E$ từ ví dụ \ref{ex:ex4}
	$$ E: Y^2 = X^3 + 3X + 8 \ \ \ \ \text{trên trường}\  \mathbb{F}_{13} $$
\end{example}
Áp dụng thuật toán cộng (\ref{th:th2}) để cộng $P(9,7)$ và $Q(1,8)$, trước hết, ta tính hệ số góc của $L$:
$$\lambda  =\frac{y_2-y_1}{x_2-x_1} = \frac{8-7}{1-9} = \frac{1}{-8} = \frac{1}{5} = 8$$

vì các tính toán \footnote{Đây là lúc thích hợp để hiểu rằng, $\frac{1}{5}$ chỉ là \textit{kí hiệu} cho một nghiệm của phương trình $5x=1$.
	Để gán một giá trị cho $\frac{1}{5}$, bạn phải biết giá trị đang ở trường nào. Trong trường $\mathbb{Q}$, giá trị của $\frac{1}{5}$ là một số bình thường,
	nhưng ở trường $\mathbb{F}_{13}$ giá trị của $\frac{1}{5}$ là $8$, trong khi ở trường $\mathbb{F}_{11}$, giá trị của $\frac{1}{5}$ là $9$.
	Còn ở trường $\mathbb{F}_5$, kí hiệu $\frac{1}{5}$ không được gán giá trị nào.} đang được thực hiện trên trường $\mathbb{F}_{13}$ nên $-8 = 5$ và $\frac{1}{5} = 5^{-1} = 8$.
Tiếp tục, ta tính
$$\nu = y_1 - \lambda x_1 = 7 - 8 \centerdot 9 = -65 = 0.$$
Cuối cùng:
$$
	\begin{array}{rcl}
		x_3 & = & \lambda^2 - x_1 - x_2 = 64 -9 -1 = 54 = 2,         \\
		y_3 & = & -(\lambda x_3 + \nu) = -8 \centerdot 2 = -16 = 10.
	\end{array}
$$
Và ta hoàn thành việc tính toán
$$P+Q = (1,8) + (9,7) = (2,10) \in E(\mathbb{F}_{13})$$

Tương tự, ta dùng thuật toán cộng để cộng điểm $P = (9,7)$ với chính nó. Lưu ý ta vẫn đang thực hiện tính toán trên trường $\mathbb{F}_{13}$, ta có:

$$
	\begin{array}{rcl}
		\lambda & = & \frac{3x_1^2+A}{2y_1} = \frac{3\centerdot 9^ 2 + 3}{2 \centerdot 7} = \frac{246}{14} = 1 \\
		\nu     & = & y_1-\lambda x_1 = 7 - 1 \centerdot 9 = 11 .
	\end{array}
$$

sau đó
$$
	\begin{array}{rcl}
		x_3 & = & \lambda^2 - x_1 - x_2 = 1 - 9 - 9 = 9          \\
		y_3 & = & -(\lambda x_3 + \nu) = -1 \centerdot 9 -11 =6.
	\end{array}
$$

nên $P+P = (9,7) + (9,7) = (9,6) \in E(\mathbb{F}_{13})$. Theo cách đó, ta có thể cộng mọi cặp điểm trong $E(\mathbb{F}_{13})$, kết quả được thể hiện ở \hyperref[fg:tb1]{Bảng 1}


\begin{figure}[H]
	\label{fg:tb1}
	\caption{Phép cộng $E: Y^2 = X^3 + 3X + *$ trên trường $\mathbb{F}_{13}$}
	$
		\begin{array}{|c||c|c|c|c|c|c|c|c|c|}
			\hline
			            & \mathcal{O} & (1,5)       & (1,8)       & (2,3)   & (2,10)      & (9,6)       & (9,7)       & (12,2)      & (12,11)     \\
			\hline
			\hline
			\mathcal{O} & \mathcal{O} & (1,5)       & (1,8)       & (2,3)   & (2,10)      & (9,6)       & (9,7)       & (12,2)      & (12,11)     \\
			\hline
			(1,5)       & (1,5)       & (2,10)      & \mathcal{O} & (1,8)   & (9,7)       & (2,3)       & (12,2)      & (12,11)     & (9,6)       \\
			\hline
			(1,8)       & (1,8)       & \mathcal{O} & (2,3)       & (9,6)   & (1,5)       & (12,11)     & (2,10)      & (9,7)       & (12,2)      \\
			\hline
			(2,3)       & (2,3)       & (1,8)       & (9,6)       & (12,11) & \mathcal{O} & (12,2)      & (1,5)       & (2,10)      & (9,7)       \\
			\hline
			(9,6)       & (9,6)       & (2,3)       & (12,11)     & (12,2)  & (1,8)       & (9,7)       & \mathcal{O} & (1,5)       & (2,10)      \\
			\hline
			(9,7)       & (9,7)       & (12,2)      & (2,10)      & (1,5)   & (12,11)     & \mathcal{O} & (9,6)       & (2,3)       & (1,8)       \\
			\hline
			(12,2)      & (12,2)      & (12,11)     & (9,7)       & (2,10)  & (9,6)       & (1,5)       & (2,3)       & (1,8)       & \mathcal{O} \\
			\hline
			(12,11)     & (12,11)     & (9,6)       & (12,2)      & (9,7)   & (2,3)       & (2,10)      & (1,8)       & \mathcal{O} & (1,5)       \\
			\hline
		\end{array}
	$
\end{figure}


Tập điểm trong $E(\mathbb{F}_{p})$ là tập hữu hạn. Chính xác hơn, có  $p$ cách chọn $X$ và mỗi cách chọn $X$, phương trình
$$Y^2 = X^3 + AX + B$$
cho nhiều nhất 2 nghiệm $Y$. Thêm điểm $\mathcal{O}$, $\#E(\mathbb{F}_{p})$ có tối đa $2p+1$ điểm.

Khi gán giá trị cho $X$, có ba trường hợp xảy ra với đại lượng
$$X^3 + AX + B$$

Thứ nhất, nó là thặng dư bình phương và có hai Modular square root, ta được hai điểm thuộc $E(\mathbb{F}_{p})$, trường hợp này xảy ra khoảng 50\%.
Thứ hai, nó không là thặng dư bình phương, ta bỏ qua $X$, trường hợp này cũng chiếm khoảng 50\%.
Thứ ba, $X^3 + AX + B = 0$, ta được một điểm thuộc $E(\mathbb{F}_{p})$, trường hợp này rất hiếm xảy ra \footnote{Phép đồng dư $X^3 + AX + B \equiv 0 (\mod p)$ có nhiều nhất ba nghiệm, và nếu $p$ lớn, tỉ lệ chọn ngẫu nhiên một trong số chúng là rất nhỏ.}.
Theo đó, số phần tử của $E(\mathbb{F}_{p})$ xấp xỉ
$$ \#E(\mathbb{F}_{p}) \approx 50\% \centerdot (2p + 1) = p + 1.$$

Một định lý nổi tiếng của Hasse, sau này được Weil và Deligne tổng quát hóa rộng rãi,
nói rằng điều này đúng với các dao động ngẫu nhiên

\begin{theorem}[Hasse]
	\label{th:th4}
	Cho $E$ là đường cong elliptic trên trường $F_{p}$. Thì
	$$ \#E(\mathbb{F}_{p}) = p + 1  -t_p \  \text{với $t_p$ thỏa mãn} \ \ |t_p| \leq 2\sqrt{p}.$$
\end{theorem}

\begin{definition}
	Đại lượng $t_p = p+1 - \#E(\mathbb{F}_p)$ ở định lý \ref{th:th4} được gọi là \textit{dấu vết của Frobenius} trên $E/\mathbb{F}_p$.
\end{definition}
$t_p$ xuất hiện dưới dạng dấu vết của một ma trận 2x2 có vai trò như phép biến đổi tuyến tính trên một không gian vector 2 chiều liên kết với  $E/\mathbb{F}_p$.

\begin{example}
	\label{ex:ex6}
	Cho phương trình $E$:
	$$E: Y^2 = X^3 + 4X + 6$$
\end{example}

Chúng ta có thể coi $E$ là đường cong elliptic trên trường $F_{p}$ cho các trường hữu hạn $\mathbb{F}_{p}$ khác nhau và đếm số điểm thuộc $E(\mathbb{F}_p).$
\hyperref[fg:tb2]{Bảng 2} liệt kê các kết quả với những số nguyên tố đầu tiên, cùng giá trị của $t_p$ để so sánh với giá trị của $2\sqrt{p}$.

\begin{figure}[H]
	\label{fg:tb2}
	\caption{Số điểm và dấu vết Frobenius của $E:Y^2 = X^3 + 4X + 6$}
	$$
		\begin{array}{|c|c|c|c|}
			\hline
			p  & \#E(\mathbb{F}_p) & t_p & 2\sqrt{p} \\
			\hline
			\hline
			3  & 4                 & 0   & 3.46      \\
			\hline
			5  & 8                 & -2  & 4.47      \\
			\hline
			7  & 11                & -3  & 5.29      \\
			\hline
			11 & 16                & -4  & 6.63      \\
			\hline
			13 & 14                & 0   & 7.21      \\
			\hline
			17 & 15                & 3   & 8.25      \\
			\hline
		\end{array}
	$$
\end{figure}

\begin{remark}
	Định lý \hyperref[th:th4]{Hasse} cho ta một giới hạn của $\#E(\mathbb{F}_p)$, nhưng không cung cấp một phương pháp để tính giá trị này.
	Về nguyên tắc, để tìm $\#E(\mathbb{F}_p)$, ta có thể thay từng giá trị của $X$ rồi kiểm tra giá trị của $X^3 + AX + B$ dựa vào bảng thặng dư bình phương $p$, nhưng độ phức tạp thời gian là $O(p)$, rất kém hiệu quả.
	Schoof \cite{schoof1985elliptic} đã tìm ra một phương pháp tốt hơn để tính $\#E(\mathbb{F}_p)$ trong thời gian $O(\log^6(p))$.
	Nghĩa là ông ấy tìm được một thuật toán có thời gian đa thức. Thuật toán của Schoof được cải thiện bởi Elkies và Atkin, với tên gọi là \textit{thuật toán SEA} \cite{schoof1995counting}.
\end{remark}