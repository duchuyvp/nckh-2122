\section{Đường cong elliptic trên trường hữu hạn}

Trong phần trước, chúng ta đã phát triển lý thuyết về đường cong elip về mặt hình học.
Tuy nhiên, để áp dụng lý thuyết về đường cong elliptic vào mật mã,
chúng ta cần xem xét các đường cong elliptic mà các điểm của nó có tọa độ trong một trường hữu hạn $F_p$.
Các ứng dụng về mật mã của đường cong Elliptic đa số chỉ sử dụng các đường
cong trên trường hữu hạn.

\subsection{Sơ bộ}
\begin{definition}
	Trường là một tập hợp K có nhiều hơn một phần tử, được định nghĩa hai phép toán cộng và nhân,
	ký hiệu bởi dấu $(+)$ và dấu $(.)$. Trường thỏa mãn các tính chất của số học.
\end{definition}
Các tính chất số học:
TODO:
\begin{enumerate}
	\item Tính kết hợp
	\item Tính giao hoán
	\item Đơn vị cộng và đơn vị nhân
	\item Nghịch đảo phép cộng
	\item Nghịch đảo phép nhân
	\item Tính phân phối
\end{enumerate}

\begin{definition}
	Trường hữu hạn (còn gọi là trường Galois) là những trường có hữu hạn số phần tử.
	Bậc của một trường hữu hạn là số phần tử của nó, là số nguyên tố hoặc lũy thừa nguyên tố.
\end{definition}
Trường hữu hạn là cơ bản trong một số lĩnh vực toán học và khoa học máy tính,
bao gồm lý thuyết số, hình học đại số, lý thuyết Galois, hình học hữu hạn, mật mã và lý thuyết mã hóa.

\begin{definition}[Bình phương modulo]
	Cho số nguyên dương $m \geq 2$. Số nguyên $a$ được gọi là \textit{bình phương modulo $m$} nếu $\gcd(a,m) = 1$ và phương trình
	$$x^2 \equiv a (\mod{m})$$
	có nghiệm
\end{definition}

Xét $F_p$ là một trường hữu hạn (hữu hạn số phần tử nguyên dương):
$$F_p = \{0, 1, 2, \ldots p-1\}$$
Với $p$ là một số nguyên tố. $F_p$ giống như cách viết $Z/mZ$ là vành các số nguyên modulo $m$.
\subsection{Đường cong elliptic trên trường hữu hạn}

Ta định nghĩa một đường cong elliptic $E$ trên trường hữu hạn $\mathbb{F}_q$ là một phương trình có dạng:
$$E: Y^2 = X^3 + AX + B\ \text{với các hằng số}\ A, B \in F_p\ \text{thỏa mãn}\ 4A^3 + 27B^2 \neq 0$$


Tập hợp các điểm trên $E$ có toạ độ thuộc $\mathbb{F} _p$ được kí hiệu bởi
$$E(F_p) = \{(x, y) : x, y \in \mathbb{F}_p\ \text{thỏa mãn}\ y^2 = x^3 + A x + B\} \cup \mathcal{O}$$

\begin{remark}
	Vì một vài lí do mà chúng tôi sẽ giải thích ở phần sau, ở đây, chúng tôi thêm điều kiện $p \geq 3$.
	Những đường cong Elliptic trên trường $\mathbb{F}_2$ có vai trò quan trọng trong mật mã, nhưng chúng rất phức tạp, nên chúng ta sẽ thảo luận về chúng ở phần 7.
\end{remark}

\begin{example}
	\label{ex:ex4}
	Xem xét đường cong elliptic
	$$E: Y^2 = X^3 + 3X + 8 \ \ \ \ \text{trên trường}\  F_{13}\ $$
\end{example}

Ta tìm các điểm thuộc $E(\mathbb{F}_{13})$ bằng cách thay tất cả giá trị của $X = 0,1,2,\ldots,12$ và kiểm tra với mỗi giá trị của $X$, $X^3 + 3X + 8$ có là bình phương modulo của 13 hay không.
Ví dụ, thay $X = 0$, ta có $X^3+3X+8 = 8$ và 8 không phải bình phương modulo của 13.
Tiếp theo, thay $X = 1$, ta được $X^3+3X+8 = 12$ và 12 là bình phương modulo của 13. Nó có 2 nghiệm
$$5^2 \equiv 12 (\mod{13}) \ \ \ \ \text{và} \ \ \ \ 8^2 \equiv 12 (\mod 13)$$.

Ta tìm được 2 điểm $(1,5)$ và $(1,8)$ thuộc $E(\mathbb{F}_{13})$. Tiếp tục theo cách này, ta kết thúc với tập hoàn chỉnh gồm 9 điểm:
$$E(\mathbb{F}_{13}) = \{(1,5), (1,8), (2,3), (2,10), (9,6), (9,7), (12,2), (12,11)\}.$$

Cho $P_1 = (x_1, y_1)$ và $P_2 = (x_2, y_2)$ thuộc $E(\mathbb{F}_{p})$. Ta định nghĩa tổng $P_1+P_2$ có tọa độ $(x_3, y_3)$ thu được bằng cách áp dụng thuật toán cộng (\ref{th:th2}).
Vì tọa độ các độ các điểm đó nằm trong trường $\mathbb{F}_p$, ta thu được $(x_3, y_3)$ có tọa độ trong trường $\mathbb{F}_p$. Nhưng điều này vẫn chưa đủ chỉ ra
$(x_3, y_3)$ có thể thuộc $E(\mathbb{F}_{p})$ hay không.

\begin{theorem}
	\label{th:th3}
	Cho $E$ là đường cong elliptic trên $\mathbb{F}_p$ và $P$ và $Q$ là 2 điểm thuộc $E(\mathbb{F}_{p})$.
	\begin{itemize}
		\item Thuật toán cộng đường cong elliptic áp dụng cho $P$ và $Q$ (\ref{th:th2}) đưa ra một điểm trong $E(\mathbb{F}_{p})$.
		      Điểm này được kí hiệu bởi $P + Q$.
		\item Luật cộng trên $E(\mathbb{F}_{p})$ thỏa mãn tất cả các tính chất được liệt kê ở định lý \ref{th:th1}.
		      Nói cách khác, luật cộng này làm cho $E(\mathbb{F}_{p})$ thành nhóm hữu hạn.
	\end{itemize}
\end{theorem}

\begin{proof}

\end{proof}

\begin{example}
	\label{ex:ex5}
	Tiếp tục với đường cong $E$ từ ví dụ \ref{ex:ex4}
	$$ E: Y^2 = X^3 + 3X + 8 \ \ \ \ \text{trên trường}\  F_{13} $$
\end{example}