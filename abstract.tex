\documentclass[12pt]{article}


% \usepackage{vntex}
\usepackage[english, vietnamese]{babel}
% \usepackage[left=2.00cm, right=2.00cm, top=2.00cm, bottom=2.00cm]{geometry}

\usepackage{enumerate}
\usepackage{amsmath}
\usepackage{amsfonts}
\usepackage{amssymb}
\usepackage{color}

\usepackage{tikz}
\usepackage[utf8]{inputenc}
% \usepackage[T5]{fontenc}


\usepackage{amsthm}
\usepackage[unicode]{hyperref}

\usepackage{graphicx}
\usepackage{enumitem}
\usepackage{a4wide}
\usepackage{epsfig}
\usepackage{latexsym}
\usepackage{array}
\usepackage{hhline}
\usepackage{fancyhdr}
\usepackage[normalem]{ulem}
\usepackage[makeroom]{cancel}
\usepackage{longtable}
\usepackage{amscd}
\usepackage{diagbox}
\usepackage{booktabs}
\usepackage{alltt}
\usepackage[framemethod=tikz]{mdframed}
\usepackage{subcaption}
\usepackage{caption}
\usepackage{listings}
\usepackage{xcolor}
\usepackage{lipsum}
\usepackage{setspace}
% \usepackage{titling}
\usepackage{multicol}
% \usepackage{indentfirst}
\usepackage{float}
\usepackage{titlesec}

\usetikzlibrary{decorations}
\usetikzlibrary{decorations.pathreplacing}
\usetikzlibrary{decorations.pathreplacing,calligraphy}
\usetikzlibrary{arrows.meta}
\usetikzlibrary{quotes}
\usetikzlibrary{intersections}
\usetikzlibrary{calc}

% \setstretch{1.25}

\setlength{\topmargin}{-0.6in}
\setlength{\textheight}{9in}
\setlength{\oddsidemargin}{0.125in}
\setlength{\textwidth}{6.25in}

% \setcounter{MaxMatrixCols}{30}

% \newtheorem{acknowledgement}[theorem]{Acknowledgement}
% \newtheorem{algorithm}[theorem]{Algorithm}
% \newtheorem{axiom}[theorem]{Axiom}
% \newtheorem{case}[theorem]{Case}
% \newtheorem{claim}[theorem]{Claim}
% \newtheorem{conclusion}[theorem]{Conclusion}
% \newtheorem{condition}[theorem]{Condition}
% \newtheorem{conjecture}[theorem]{Conjecture}
\newtheorem{corollary}{Hệ quả}
% \newtheorem{criterion}[theorem]{Criterion}
% \newtheorem{definition}[theorem]{Definition}
% \newtheorem{example}[theorem]{Example}
% \newtheorem{exercise}[theorem]{Exercise}
% \newtheorem{lemma}[theorem]{Lemma}
% \newtheorem{notation}[theorem]{Notation}
% \newtheorem{problem}[theorem]{Problem}
% \newtheorem{proposition}[theorem]{Proposition}
% \newtheorem{remark}[theorem]{Remark}
% \newtheorem{solution}[theorem]{Solution}
% \newtheorem{summary}[theorem]{Summary}
\newtheorem{theorem}{Định lý}
% \newtheorem*{recap}{Recap}
% \newenvironment{proof}[1][Proof]{\noindent\textbf{#1.} }
% {\hfill \ \rule{0.5em}{0.5em}}
% \newmdenv[linecolor=black,skipabove=\topsep,skipbelow=\topsep,
% leftmargin=-5pt,rightmargin=-5pt,
% innerleftmargin=5pt,innerrightmargin=5pt]{textbox}

\definecolor{dkgreen}{rgb}{0,0.6,0}
\definecolor{gray}{rgb}{0.5,0.5,0.5}
\definecolor{mauve}{rgb}{0.58,0,0.82}

\lstset{
%   frame=tb,
    language=Java,
    aboveskip=3mm,
    belowskip=3mm,
    showstringspaces=false,
    columns=flexible,
    basicstyle={\small\ttfamily},
    numbers=none,
    numberstyle=\tiny\color{gray},
    keywordstyle=\color{blue},
    commentstyle=\color{dkgreen},
    stringstyle=\color{mauve},
    breaklines=true,
    breakatwhitespace=true,
    tabsize=3
}

\title{Đồ thị con dày đặc của đồ thị phân cực và số chu trình 4 đỉnh cực đại}
\author{
    Michael Tait\thanks{Department of Mathematics, University of California San Diego, \url{mtait@math.ucsd.edu}}
    \and
    Craig Timmons\thanks{Department of Mathematics and Statistics, California State University Sacramento, \mbox{\url{craig.timmons@csus.edu}}}
}
\date{}

\begin{document}

\maketitle
\vspace{-5mm}

\begin{abstract}
    Trong nội dung bài báo, chúng tôi sẽ chứng minh rằng với mọi \mbox{$m \in \{1,2, \dots , q +1 \}$}, nếu $G$ là một đồ thị phân cực của một mặt phẳng ánh xạ bậc $q$ chứa một hình oval, thì $G$ chứa một đồ thị con với $m + \binom{m}{2}$ đỉnh và $m^2+\frac{m^4}{8q} - O ( \frac{m^4}{q^{3/2} } +m )$ cạnh. As an application, ta sẽ đưa ra một cận dưới tốt nhất cho số Tur\'{a}n  $\mathrm{ex}(n, C_4)$ với một giá trị chính xác của $n$. Cụ thể hơn, chúng tôi bác bỏ phỏng đoán của Abreu, Balbuena, và Labbate về $\mathrm{ex}(q^2-q-2, C_4)$ với $q$ là lũy thừa của $2$.
\end{abstract}


Michael Tait và  Craig Timmons đưa ra các khái niệm về $F$-\emph{free}, số Tur\'{a}n, đồ thị phân cực, hình học hữu hạn, điểm phân cực và hình oval:

Cho $F$ là một đồ thị:
\begin{itemize}
    \item  Đồ thị $G$ được gọi là $F$-\emph{free}  nếu $F$ không phải là đồ thị con của $G$.
    \item $\textup{ex}(n , F)$ được định nghĩa là số Tur\'{a}n của $F$, là số cạnh tối đa của 1 đỉnh trong một đồ thị $F$-\emph{free} $n$ đỉnh
    \item $\textup{Ex}(n  , F)$ là họ đồ thị $n$ đỉnh là $F$-\emph{free} và có các cạnh ex(n, F). Các đồ thị trong họ $\textup{Ex}(n  , F)$ được gọi là \emph{đồ thị phân cực}.
    \item Cho $\mathcal{P}$ và $\mathcal{L}$ là các tập rời rạc, hữu hạn, và cho $\mathcal{I}\subset \mathcal{P}\times \mathcal{L}$. Bộ ba $(\mathcal{P}, \mathcal{L}, \mathcal{I})$ là một hình học hữu hạn. Các phần tử của $\mathcal{P}$ được gọi là điểm, các phần tử của $\mathcal{L}$ được gọi là đoạn thẳng. Một cực của hình học là một phép chiếu từ $\mathcal{P}\cup \mathcal{L}$ đến $\mathcal{P}\cup \mathcal{L}$ để gửi điểm tới đường, gửi đường tới điểm.
    \item Cho một hình học hữu hạn $(\mathcal{P}, \mathcal{L}, \mathcal{I})$ và một cực $\pi$, đồ thị phân cực $G_\pi$ là đồ thị có tập đỉnh $V(G_\pi) = \mathcal{P}$ và tập cạnh
    $$ E(G_\pi) = \{\{p,q\}: p,q\in \mathcal{P}, (p, \pi(q))\in \mathcal{I}\} $$
    \item Điểm phân cực là điểm mà $(p, \pi(p))\in \mathcal{I}$.
    \item Hình oval trong mặt ánh xạ bậc $q$ là tập hợp của $q+1$ điểm trong đó không có 3 điểm nào thẳng hàng.


\end{itemize}

Hiện nay việc xác định số cạnh tối đa của các đồ thị $F$-\emph{free}  khác nhau là một trong những vấn đề được nghiên cứu nhiều nhất trong lý thuyết đồ thị phân cực. Đặc biệt là $F = C_4$ - chu trình bốn đỉnh.

Michael Tait và  Craig Timmons đã phát biểu định lý

Định lý 1.1:

\begin{theorem}\label{crass subgraph}
     Gọi $\Pi$ là một mặt phẳng ánh xạ bậc $q$, chứa hình oval và có một phân cực $\pi$. Nếu $m \in \{1,2, \dots , q + 1 \}$, thì đồ thị phân cực $G_{ \pi}$ chứa một đồ thị con có nhiều nhất $m + \binom{m}{2}$ đỉnh và có ít nhất $$ 2 \binom{m}{2} + \frac{m^4}{8q} - O \left( \frac{m^4}{q^{3/2} } + m \right) $$ cạnh.
\end{theorem}
Bằng cách xem xét các đồ thị con của $ER_q$, Abreu, Balbuena và Labbate đã chứng minh rằng
$$ \mathrm{ex}(q^2-q-2, C_4) \geq \frac{1}{2}q^3 - q^2 $$
trong đó $q$ là lũy thừa của 2.

Tuy nhiên, bằng cách sử dụng định lý \ref{crass subgraph}, Michael Tait và  Craig Timmons đã bác bỏ phỏng đoán trên và đưa ra một kết quả mới

\begin{corollary}\label{speculate}
    Nếu $q$ là lũy thừa của số nguyên tố, thì
    $$ \mathrm{ex}(q^2 - q - 2, C_4) \geq \frac{1}{2}q^3 - q^2 + \frac{3}{2}q - O\left(q^{1/2}\right) $$
\end{corollary}

\end{document}