
\section{Đường cong Elliptic}
Một \textit{đường cong Elliptic} là tập nghiệm của một phương trình có dạng
$$Y^2 = X^3 + AX + B$$
Các phương trình thuộc loại này được gọi là \textit{phương trình Weierstrass} sau khi ông đã nghiên cứu chúng trong suốt thể kỉ XIX. Hai ví dụ cho đường cong elliptic:
$$ E_1: Y^2=X^3-3X+3 $$ và $$ E_2: Y^2=X^3-5X+5 $$ được minh họa ở \hyperref[fg:fg1]{Hình 1}

\begin{figure}[H]
	\caption{Hình 1}
	\label{fg:fg1}
	\begin{minipage}{0.4\textwidth}
		\centering
		\begin{tikzpicture}[scale = 0.5, line cap=round,line join=round,>=triangle 45,x=1cm,y=1cm]
			\begin{axis}[
					x=1cm,y=1cm,
					axis lines=middle,
					xmin=-5.912600967627929,
					xmax=5.112504977805202,
					ymin=-5.000000419005899,
					ymax=5.000000419005899,
					xtick={-5,-4,...,5},
					ytick={-5,-4,...,5},]
				\clip(-5.912600967627929,-5.000000419005899) rectangle (10.112504977805202,5.000000419005899);
				\draw[line width=2pt,color=blue,smooth,samples=100,domain=-2.1038033040947663:10.112504977805202] plot(\x,{sqrt((\x)^(3)-3*(\x)+3)});
				\draw[line width=2pt,color=blue,smooth,samples=100,domain=-2.1038033040947663:10.112504977805202] plot(\x,{0-sqrt((\x)^(3)-3*(\x)+3)});
			\end{axis}
		\end{tikzpicture}
	\end{minipage}
	\hfill
	\begin{minipage}{0.4\textwidth}
		\centering
		\begin{tikzpicture}[scale = 0.5, line cap=round,line join=round,>=triangle 45,x=1cm,y=1cm]
			\begin{axis}[
					x=1cm,y=1cm,
					axis lines=middle,
					xmin=-5.912600967627929,
					xmax=5.112504977805202,
					ymin=-5.000000419005899,
					ymax=5.000000419005899,
					xtick={-5,-4,...,5},
					ytick={-5,-4,...,5},]
				\clip(-5.912600967627929,-5.000000419005899) rectangle (10.112504977805202,5.000000419005899);
				\draw[line width=2pt,color=blue,smooth,samples=100,domain=-2.791287717641796:0.9999998864501113] plot(\x,{sqrt(abs((\x)^(3)-6*(\x)+5))});
				\draw[line width=2pt,color=blue,smooth,samples=100,domain=1.791293439719924:10.112505461619168] plot(\x,{sqrt(abs((\x)^(3)-6*(\x)+5))});
				\draw[line width=2pt,color=blue,smooth,samples=100,domain=-2.791287717641796:0.9999998864501113] plot(\x,{0-sqrt(abs((\x)^(3)-6*(\x)+5))});
				\draw[line width=2pt,color=blue,smooth,samples=100,domain=1.791293439719924:10.112505461619168] plot(\x,{0-sqrt(abs((\x)^(3)-6*(\x)+5))});
			\end{axis}
		\end{tikzpicture}
	\end{minipage}
\end{figure}


Một điều tuyệt vời của đường cong elliptic là có một cách tự nhiên để chọn hai điểm trên đường cong và
``cộng'' chúng để tạo ra điểm thứ ba. Phép ``cộng'' được chúng tôi nhắc đến ở đây là một phép toán kết hợp hai điểm theo cách tương tự với phép cộng thông thường
ở một vài khía cạnh (có tính chất giao hoán, kết hợp và có cách nhận dạng), nhưng rất khác ở những phần còn lại. Một trong những cách đơn giản để miêu tả "luật cộng" là sử dụng hình học.

Cho $P$ và $Q$ là hai điểm trên đường cong elliptic $E$, như minh họa ở \hyperref[fg:fg2]{Hình 2}. Ta bắt đầu vẽ một đường thẳng $L$ đi qua $P$ và $Q$. Đường thẳng
$L$ sẽ cắt $E$ tại ba điểm $P$, $Q$ và một điểm $R$ thứ ba. Ta lấy đối xứng điểm $R$ qua trục $Ox$ để được điểm $R'$. Điểm $R'$ này gọi là \textit{tổng của $P$ và $Q$},
phép ``cộng'' này không giống phép cộng thông thường. Ta biểu thị phép ``cộng'' này bằng kí hiệu $\oplus$. Ta viết
\begin{equation}
	P \oplus Q = R'
\end{equation}

\begin{example}
	\label{ex:ex1}
	Cho đường cong elliptic E:
	\begin{equation}
		\label{eq:curve}
		Y^2 = X^3 -15X + 18
	\end{equation}
\end{example}
Điểm $P = (7,16)$ và $Q = (1,2)$ nằm trên $E$. Đường thẳng $L$ nối $P$ và $Q$ có phương trình
\begin{equation}
	\label{eq:line}
	L: Y = \frac{7}{3}X - \frac{1}{3}
\end{equation}
Để tìm giao điểm của $E$ và $L$, ta thay $Y$ ở phương trình \eqref{eq:line} vào phương trình \eqref{eq:curve} để tìm $X$. Ta có
\begin{center}

	$
		\begin{array}{rcl}
			(\frac{7}{3}X - \frac{1}{3})^2                & = & X^3 -15X+18                                        \\
			\frac{49}{9}X^2 - \frac{14}{9}X + \frac{1}{9} & = & X^3 -15X+18                                        \\
			0                                             & = & X^3 -\frac{49}{9}X^2 -\frac{121}{9}X+\frac{161}{9} \\
		\end{array}
	$
\end{center}

\begin{figure}[H]
	\caption{Hình 2}
	\label{fg:fg2}
	\centering
	\begin{tikzpicture}[line cap=round,line join=round,>=triangle 45,x=1cm,y=1cm]
		\clip(-5.736430725533388,-5.733053262983473) rectangle (5.923722428745249,5.872549467181578);
		\draw[line width=0.8pt,color=blue,smooth,samples=100,domain=-2.103803291543122:5.923722428745249] plot(\x,{sqrt(abs((\x)^(3)-3*(\x)+3))});
		\draw[line width=0.8pt,color=blue,smooth,samples=100,domain=-2.103803291543122:5.923722428745249] plot(\x,{0-sqrt(abs((\x)^(3)-3*(\x)+3))});
		\draw [line width=0.8pt,domain=-5.736430725533388:5.923722428745249] plot(\x,{(--2.3486301286669415-0.40680839258599555*\x)/1.3559822370127241});
		\draw [line width=0.8pt,dotted] (1.4459883083356893,-1.9982396827947813)-- (1.4459883083356893,1.9982396827947813);
		\begin{scriptsize}
			\draw[color=black] (-1.9451762496392822,0.28794979855338965) node {$E$};
			\draw [fill=black] (0,1.7320508075688772) circle (2pt);
			\draw[color=black] (0.2232031088756924,1.9926505521028977) node {$Q$};
			\draw [fill=black] (-1.3559822370127241,2.1388592001548727) circle (2pt);
			\draw[color=black] (-1.1405574939639143,2.4017787329547793) node {$P$};
			\draw[color=black] (-5.463678604965468,2.8700487609709) node {$L$};
			\draw [fill=black] (1.4459883083356893,1.2982396827947813) circle (2pt);
			\draw[color=black] (1.777890196112844,1.5698847652226198) node {$\ R$};
			\draw [fill=black] (1.4459883083356893,-1.2982396827947813) circle (2pt);
			\draw[color=black] (2.4324952854758553,-1.0348979862010284) node {$R' = P \oplus Q$};
		\end{scriptsize}
	\end{tikzpicture}
\end{figure}
Thông thường, việc tìm nghiệm của phương trình bậc ba không đơn giản, nhưng ta đã biết trước 2 giao điểm của $L$ và $E$ là $P$ và $Q$, nên rõ ràng phương trình trên
có 2 nghiệm $X=1$ và $X=7$. Từ đó, ta dễ dàng tìm được nghiệm còn lại
$$X^3 -\frac{49}{9}X^2 -\frac{121}{9}X+\frac{161}{9} = (X-7) \centerdot  (X-1) \centerdot (X+\frac{23}{9})$$

Thay $X=-\frac{23}{9}$ vào phương trình \eqref{eq:line} ta được điểm $R = (-\frac{23}{9}, -\frac{170}{27})$. Cuối cùng, lấy đối xứng qua trục $Ox$ ta được
$$P \oplus Q = (-\frac{23}{9}, \frac{170}{27})$$

Điều gì xảy ra khi ta cộng điểm $P$ với chính nó?
Khi điểm $Q$ tiến dần đến $P$, đường thẳng $L$ sẽ trở thành tiếp tuyến của $E$ tại $P$. Vậy, để cộng điểm $P$ với chính nó, ta đơn giản chỉ cần lấy $L$ là
tiếp tuyến của $E$ tại $P$, minh họa ở \hyperref[fg:fg3]{Hình 3}. Khi đó $L$ giao $E$ tại $P$ và một điểm $R$ khác, điểm $P$ được tính 2 lần.
\begin{figure}[H]
	\caption{Hình 3}
	\label{fg:fg3}
	\centering
	\begin{tikzpicture}[line cap=round,line join=round,>=triangle 45,x=1cm,y=1cm]
		\clip(-5.736430725533388,-5.733053262983473) rectangle (5.923722428745249,5.872549467181578);
		\draw[line width=0.8pt,color=blue,smooth,samples=100,domain=-2.103803291543122:5.923722428745249] plot(\x,{sqrt(abs((\x)^(3)-3*(\x)+3))});
		\draw[line width=0.8pt,color=blue,smooth,samples=100,domain=-2.103803291543122:5.923722428745249] plot(\x,{0-sqrt(abs((\x)^(3)-3*(\x)+3))});
		\draw [line width=0.8pt,domain=-5.736430725533388:5.923722428745249] plot(\x,{(-0.011820467984078764-0.0023798531917433863*\x)/-0.0040177629872759635});
		\draw [line width=0.8pt,dotted] (3.06684049847989,-5.458642606222206)-- (3.06684049847989,5.458642606222206);
		\begin{scriptsize}
			\draw[color=black] (-1.9451762496392822,0.28794979855338965) node {$E$};
			\draw [fill=black] (-1.36,2.1364793469631294) circle (2pt);
			\draw [fill=black] (-1.3559822370127241,2.1388592001548727) circle (2pt);
			\draw[color=black] (-1.1405574939639143,2.4017787329547793) node {$P$};
			\draw[color=black] (-5.463678604965468,-0.10754077627009614) node {$L$};
			\draw [fill=black] (3.06684049847989,4.758642606222206) circle (2pt);
			\draw[color=black] (3.400765313491976,5.020199090406824) node {$\ R$};
			\draw [fill=black] (3.06684049847989,-4.758642606222206) circle (2pt);
			\draw[color=black] (4.0553704028549875,-4.4988499174136285) node {$R' = P \oplus P = 2P$};
		\end{scriptsize}
	\end{tikzpicture}
\end{figure}
\begin{example}
	\label{ex:ex2}
	Tiếp tục với đường cong $E$ và điểm $P$ ở \hyperref[ex:ex1]{ví dụ 1}, ta tính $P \oplus P$.
\end{example}
Ta tìm độ dốc tại $P$ của $E$ bằng cách đạo hàm 2 vế phương trình \eqref{eq:curve}. Ta được
$$2\frac{dY}{dX} = 3X^2-15, \ \text{suy ra} \ \frac{dY}{dX} = \frac{3X^2-15}{2Y} $$

Thay tọa độ điểm $P = (7,16)$ ta được độ dốc $\lambda = \frac{33}{8}$, nên đường tiếp tuyến của $E$ tại $P$ có phương trình
\begin{equation}
	\label{eq:tang}
	L: Y = \frac{33}{8}X - \frac{103}{8}
\end{equation}
Tiếp theo, thay $Y$ ở phương trình \eqref{eq:tang} vào phương trình \eqref{eq:curve}:
\begin{center}

	$
		\begin{array}{rcl}
			(\frac{33}{8}X - \frac{103}{8})^2                             & = & X^3 -15X+18 \\
			X^3 - \frac{1089}{64}X^2 + \frac{2919}{32}X - \frac{9457}{64} & = & 0           \\
			(X-7)^2 \centerdot (X-\frac{193}{64})                         & = & 0           \\
		\end{array}
	$
\end{center}
Ta đã biết trước $X=7$ là nghiệm bội 2 của phương trình bậc 3 nên dễ dàng phân tích thành nhân tử và tìm được nghiệm còn lại.
Cuối cùng, thay $X=\frac{193}{64}$ vào phương trình \eqref{eq:tang} ta được $Y = -\frac{233}{512}$. Đổi dấu  $Y$ ta được
$$P \oplus P = (\frac{193}{64},\frac{233}{512})$$

Vấn đề thứ hai là khi ta cố gắng cộng điểm $P=(a,b)$ với điểm đỗi xứng của nó qua trục $Ox$ $P'=(a,-b)$.
Đường thẳng $L$ đi qua $P$ và $P'$ có phương trình $x=a$, chỉ cắt $E$ tại 2 điểm $P$ và $P'$. (\hyperref[fg:fg4]{Hình 4})
Vậy nên không có giao điểm thứ ba. Giải pháp là tạo thêm một điểm $\mathcal{O}$ ở "vô cực".
Chính xác hơn, điểm $\mathcal{O}$ không tồn tại trên mặt phẳng $ Oxy$, nhưng ta giả định nó nằm trên mọi đường thẳng đứng.
Ta có:
$$ P \oplus P' = \mathcal{O}$$

Tiếp theo, ta cần tìm cách  cộng điểm $\mathcal{O}$ với điểm $P = (a,b)$ thuộc $E$.
Đường thẳng $L$ nối $P$ với $\mathcal{O}$ là đường thẳng đứng đi qua $P$ và cắt $E$ tại $P'=(a,-b)$.
Để cộng $P$ với $\mathcal{O}$, ta lấy điểm đối xứng với $P'$ qua trục $Ox$, ta được điểm $P$.
Nói cách khác $P \oplus \mathcal{O} = P$, vậy điểm $\mathcal{O}$ có vai trò như số 0 trong phép cộng elliptic.

\begin{figure}[H]
	\caption{Hình 4}
	\label{fg:fg4}
	\centering
	\begin{tikzpicture}[line cap=round,line join=round,>=triangle 45,x=1cm,y=1cm]
		\clip(-5.083985307432931,-5.419798893648197) rectangle (4.940270977414931,5.225844038096666);
		\draw[line width=0.8pt,color=blue,smooth,samples=100,domain=-2.103803385261131:4.940270977414931] plot(\x,{sqrt(abs((\x)^(3)-3*(\x)+3))});
		\draw[line width=0.8pt,color=blue,smooth,samples=100,domain=-2.103803385261131:4.940270977414931] plot(\x,{0-sqrt(abs((\x)^(3)-3*(\x)+3))});
		\draw [->,line width=0.8pt] (-1.3559822370127241,-5.8388592001548725) -- (-1.3559822370127241,5.138859200154872);
		\begin{scriptsize}
			\draw[color=black] (-1.9653277965913738,0.24888872474237672) node {$E$};
			\draw [fill=black] (-1.3559822370127241,2.1388592001548727) circle (2pt);
			\draw[color=black] (-0.4929001609366532,2.3709827452774213) node {$P = (a,b)$};
			\draw [fill=black] (-1.3559822370127241,-2.1388592001548727) circle (2pt);
			\draw[color=black] (-0.2929001609366532,-2.408378124862309) node {$P'=(a,-b)$};
			\draw[color=black] (-0.7460030555104643,5.020669201857089) node {$\mathcal{O}$};
			\draw[color=black] (-1.1680770043461637,1.292349320475078) node {$\ L$};
		\end{scriptsize}
	\end{tikzpicture}
\end{figure}
\begin{example}
	\label{ex:ex3}
	Tiếp tục với đường cong $E$ ở \hyperref[ex:ex1]{ví dụ 1} và điểm $T = (3,0)$.
\end{example}
Chú ý điểm $T$ nằm trên $E$ và tiếp tuyến tại $T$ là đường thẳng đứng $X=3$. Vậy nếu cộng điểm $T$ với chính nó, ta được $T \oplus T = \mathcal{O}$.

\begin{definition}
	Một đường cong elliptic $E$ là tập nghiệm của một phương trình \mbox{Weierstrass}:
	$$E : Y^2 = X^3 + AX+ B$$
	cùng với một điểm $\mathcal{O}$ ở vô cùng, trong đó hằng số $A$ và $B$ thỏa mãn
	$$ 4A^3 + 27B^2 \neq 0$$
\end{definition}

\textit{Luật cộng} trên $E$ được định nghĩa như sau. Cho 2 điểm $P$ và $Q$ là 2 điểm thuộc $E$.
$L$ là đường thẳng nối $P$ và $Q$, hoặc là đường tiếp tuyến của $E$ tại $P$ nếu $P=Q$.
Khi đó, giao điểm của $E$ và $L$ là ba điểm $P$, $Q$ và $R$, với $\mathcal{O}$
được hiểu là điểm nằm trên mọi đường thẳng đứng. $R=(a,b)$, tổng của $P$ và $Q$
là điểm $R'=(a,-b)$. Tổng này được ký hiệu là $P \oplus Q$, có thể viết đơn giản $P+Q$.

Ta biểu diễn điểm đối xứng của $P$ bởi $\ominus P = (a,-b)$, hoặc  $-P$;
ta định nghĩa $P \ominus P$ (hay $P - Q$) là $P \oplus (\ominus Q)$.
Tương tự, lặp lại phép cộng nhiều lần là biểu diễn của phép nhân một điểm với một số nguyên,
$$nP = \underbrace{P + P + P + \ldots + P}_{\text{$n$ số hạng}}$$

\begin{remark}
	Tại sao cần thỏa mãn điều kiện $4A^3 + 27B^2 \neq 0$?
\end{remark}
Đại lượng $\Delta_E = 4A^3 + 27B^2$ được gọi là \textit{phân thức của $E$}.
$\Delta_E \neq 0$ là điều kiện để đa thức $X^3 + AX + B$ có 3 nghiệm phân biệt, nếu phân tích thành nhân tử $X^3 + AX + B$
ta được:
$$X^3 + AX + B = (X-e_1)(X-e_2)(X-e_3)$$
trong đó $e_1, e_2, e_3$ là các số phức, thì
$$4A^3 + 27B^2 \neq 0 \ \ \ \ \ \text{khi và chỉ khi} \ \ \ \ \ e_1, e_2, e_3 \ \text{phân biệt}$$

Phép cộng không hoàn toàn đúng đối với đường cong có $\Delta_E = 0$ nên chúng tôi thêm điều kiện $\Delta_E \neq 0$ khi nêu
khái niệm đường cong elliptic.

\begin{theorem}
	\label{th:th1}
	Cho đường cong elliptic $E$. Luật cộng trên $E$ thỏa mãn các tính chất sau:
\end{theorem}
\begin{tabular}{crclll}
	(a) \label{th:th1:a} & $P + \mathcal{O}$ & $=$ & $\mathcal{O} + P = P$ & $\forall P \in E$     & [Cộng với điểm $\mathcal{O}$] \\
	(b) \label{th:th1:b} & $P + (-P)$        & $=$ & $\mathcal{O}$         & $\forall P \in E$     & [Nghịch đảo]                  \\
	(c) \label{th:th1:c} & $(P + Q) + R$     & $=$ & $P + (Q + R)$         & $\forall P,Q,R \in E$ & [Kết hợp]                     \\
	(d) \label{th:th1:d} & $P + Q$           & $=$ & $ Q + P$              & $\forall P,Q \in E$   & [Giao hoán]                   \\
\end{tabular}

\begin{proof}
	Như đã giải thích trước đó, dễ thẩy tính \hyperref[th:th1:a]{cộng với điểm $\mathcal{O}$} và tính \hyperref[th:th1:b]{nghịch đảo}
	là đúng vì $\mathcal{O}$ nằm trên mọi đường thẳng đứng. Tính \hyperref[th:th1:d]{giao hoán} dễ dàng chứng minh vì đường thẳng qua $P$ và $Q$
	cũng là đường thẳng qua $Q$ và $P$.

	Phần còn lại cần chứng minh của định lý \ref{th:th1} là tính \hyperref[th:th1:c]{kết hợp}. Có nhiều cách để chứng minh tính \hyperref[th:th1:c]{kết hợp}, nhưng không
	cách nào trong số chúng đơn giản. Sau khi có đủ các kiến thức cần thiết
	về luật cộng trên $E$ (\ref{th:th2}), bạn đọc có thể sử dụng để tự chứng minh. Có thể tìm thấy những chứng minh rõ ràng hơn ở \cite{halmos1982graduate}, \cite{silverman2009arithmetic} hoặc \cite{silverman1992rational}
	và một vài quyển sách khác về đường cong elliptic.
\end{proof}

Tiếp theo, chúng ta sẽ chứng minh một vài công thức để dễ dàng cộng và trừ các điểm trên một đường cong elliptic.
Những công thức này sử dụng hình học giải tích, tính toán vi phân và một vài thao tác đại số cơ bản.
Chúng tôi đưa kết quả dưới dạng một định lý và đưa ra chứng minh sau đó.


\begin{theorem}[Thuật toán cộng đường cong Elliptic]
	\label{th:th2}
	Cho
	$$E:Y^2 = X^3 + AX + B$$
	là một đường cong elliptic và $P_1$  và $P_2$ là hai điểm trên $E$.
	\begin{enumerate}
		\item Nếu $P_1 = \mathcal{O}$ thì $P_1 + P_2 = P_2$.
		\item Ngược lại, nếu $P_2 = \mathcal{O}$ thì $P_1 + P_2 = P_1$.
		\item Ngược lại, viết $P_1 = (x_1, y_2)$ và $P_2 = (x_2, y_2)$.
		\item Nếu $x_1 = x_2$ và $y_1 = -y_2$ thì $P_1 + P_2 = \mathcal{O}$.
		\item Nếu không, định nghĩa $\lambda$ bởi
		      $$ \lambda = \begin{cases}
				      \frac{y_2-y_1}{x_2 - x_1} & \text{nếu } P_1 \neq P_2 \\
				      \frac{3x_1^2 + A}{2y_1}   & \text{nếu } P_1 = P_2.
			      \end{cases} $$
		      Và ta có $P_1 + P_2 = (x_3, y_3)$, trong đó:
		      $$x_3 = \lambda^2 - x_1 - x_2 \ \ \  \text{ và } \ \ \  y_3 = \lambda(x_1 - x_3) - y_1.$$
	\end{enumerate}
\end{theorem}

\begin{proof}
	Phần (1) và (2) của định lý \ref{th:th2} là đúng, và (4) là trường hợp đường thẳng qua $P_1$ và $P_2$ là đường thẳng đứng, nên $P_1+P_2 = \mathcal{O}$.
	(lưu ý, vẫn đúng với trường hợp $y_1 = y_2 = 0$.) Còn phần (5), ta để ý rằng $\lambda$ là hệ số góc của đường thẳng đi qua $P_1$ và $P_2$, và cũng là hệ số góc của tiếp tuyến tại $P_1$ nếu $P_1 = P_2$.
	Trong cả 2 trường hợp, đường thẳng $L$ đều có phương trình $Y = \lambda X + \nu $ với $\nu = y_1 - \lambda x_1$. Thế $Y$ vào phường trình đường cong $E$ ta được:
	$$(\lambda X + \nu) ^ 2  = X^3 + AX + B$$
	nên
	$$X^3 - \lambda^2X^2 + (A-2\lambda \nu)X + (B-\nu^2) = 0$$

	Phương trình trên có 2 nghiệm đã biết trước là $x_1$ và $x_2$. Ta gọi nghiệm còn lại là $x_3$, phân tích thành nhân tử ta được
	$$X^3 - \lambda^2X^2 + (A-2\lambda \nu)X + (B-\nu^2) = (X-x_1)(X-x_2)(X-x_3)$$
	Đồng nhất hệ số 2 vế, ta được $x_3 = \lambda^2 - x_1 - x_2$. Cuối cùng, để tìm được $P_1 + P_2$, ta thay $x_3$ vào phương trình $L$ để tìm giao điểm còn lại của $L$ và $E$ rồi lấy đối xứng qua $Ox$.
\end{proof}

