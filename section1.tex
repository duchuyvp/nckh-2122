
\section{Nhắc lại}
\begin{definition}
	Trường là một tập hợp K có nhiều hơn một phần tử, được định nghĩa hai phép toán cộng và nhân,
	ký hiệu bởi dấu $(+)$ và dấu $(.)$. Trường thỏa mãn các tính chất của số học.
\end{definition}
Các tính chất số học:
TODO:
\begin{enumerate}
	\item Tính kết hợp
	\item Tính giao hoán
	\item Đơn vị cộng và đơn vị nhân
	\item Nghịch đảo phép cộng
	\item Nghịch đảo phép nhân
	\item Tính phân phối
\end{enumerate}

\begin{definition}
	Trường hữu hạn (còn gọi là trường Galois) là những trường có hữu hạn số phần tử.
	Bậc của một trường hữu hạn là số phần tử của nó, là số nguyên tố hoặc lũy thừa nguyên tố.
\end{definition}
Trường hữu hạn là cơ bản trong một số lĩnh vực toán học và khoa học máy tính,
bao gồm lý thuyết số, hình học đại số, lý thuyết Galois, hình học hữu hạn, mật mã và lý thuyết mã hóa.

\begin{definition}[Bình phương modulo]
	Cho số nguyên dương $m \geq 2$. Số nguyên $a$ được gọi là \textit{bình phương modulo $m$} nếu $\gcd(a,m) = 1$ và phương trình
	$$x^2 \equiv a (\mod{m})$$
	có nghiệm
\end{definition}

\begin{definition}[Nghịch đảo modulo]
	Với một số nguyên $a$, ta gọi nghịch đảo modulo $m$ của $a$ là $a^{-1}$ là số nguyên thỏa mãn:
	$$a * a^{-1} \equiv 1 (\mod{m})$$
	Chú ý rằng không phải lúc nào $a^{-1}$ cũng tồn tại. Ví dụ với $m = 4, a = 2$, ta không thể tìm được $a^{-1}$ thỏa mãn đằng thức trên.
\end{definition}
\begin{proposition}
	\label{pr:13}
	Cho số nguyên $m \geq 1$.
	\begin{itemize}
		\item Nếu $a_1 \equiv a_2 \pmod{m}$ và $b_1 \equiv b_2 \pmod{m}$, thì
		      $$a_1 \pm b_1 \equiv a_2 \pm b_2 \pmod{m} \quad \text{ và } \quad a_1 \cdot b_1 \equiv a_2 \cdot b_2 \pmod{m}$$.
		\item Cho số nguyên $a$. Tồn tại nghịch đảo modulo $m$ của $a$ khi và chỉ khi $\gcd(a,m) = 1$
	\end{itemize}
\end{proposition}
\begin{proof}
	\textcolor{red}{Chứng minh sau}
\end{proof}

\begin{definition}[Thặng dư bình phương]
	Một số nguyên $q$ gọi là thặng dư bình phương theo modulo $m$ nếu nó đồng dư với một số chính phương theo modulo $m$.
	Nói cách khác, tồn tại số nguyên $x$ thỏa mãn:
	$$x^2 \equiv q (\mod{m})$$
	Ngược lại, $q$ được gọi là \textit{phi thặng dư bình phương}

\end{definition}

\begin{definition}[Modular square root]
	Một Modular square root $r$ của số nguyên $a$ theo modulo $m$ là một số nguyên thỏa mãn:
	$$r^2 \equiv a (\mod{m})$$
\end{definition}

Xét $F_p$ là một trường hữu hạn (hữu hạn số phần tử nguyên dương):
$$F_p = \{0, 1, 2, \ldots p-1\}$$
Với $p$ là một số nguyên tố. $F_p$ giống như cách viết $Z/mZ$ là vành các số nguyên modulo $m$.


