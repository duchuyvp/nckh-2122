\documentclass[12pt]{article}
% \usepackage{vntex}
\usepackage[utf8]{inputenc}
\usepackage[T5]{fontenc}
% \usepackage[english, vietnamese]{babel}
\usepackage{tikz}
\usepackage[left=2.00cm, right=2.00cm, top=2.00cm, bottom=2.00cm]{geometry}
\usepackage[unicode]{hyperref}
\usepackage{amsmath}
\usepackage{amssymb}
\usepackage{graphicx}
\usepackage{enumitem}
\usepackage{a4wide,amssymb,epsfig,latexsym,array,hhline,fancyhdr}
\usepackage[normalem]{ulem}
\usepackage[makeroom]{cancel}
\usepackage{amsthm}
\usepackage{multicol,longtable,amscd}
\usepackage{diagbox}
\usepackage{booktabs}
\usepackage{alltt}
\usepackage[framemethod=tikz]{mdframed}
\usepackage{caption,subcaption}
\usepackage{listings}
\usepackage{color}
\usepackage{xcolor}
\usepackage{lipsum}
\usepackage{setspace}
\usepackage{titling}
\usepackage{multicol}
\usepackage{indentfirst}
\usepackage{float}
\usepackage{titlesec}

% Without date begin
\usepackage{titling}
\date{}
% Without date end
\usetikzlibrary{decorations}
\usetikzlibrary{decorations.pathreplacing}
\usetikzlibrary{decorations.pathreplacing,calligraphy}
\usetikzlibrary{arrows.meta}
\usetikzlibrary{quotes}
\usetikzlibrary{intersections}
\usetikzlibrary{calc}

\setstretch{1.25}

\newtheorem{theorem}{Theorem}
\newtheorem{corollary}{Corollary}
\newtheorem{lemma}{Lemma}
\newtheorem*{remark}{Remark}
\newtheorem{definition}{Definition}
\newtheorem*{recap}{Recap}

\newenvironment{solution}
  {\renewcommand\qedsymbol{$\blacktriangleleft $}\begin{proof}[\textcolor{Mycolor}{Solution}]}
  {\end{proof}}

\newcounter{example}
\newenvironment{example}[1][]{\refstepcounter{example}\par\medskip
    \noindent \textcolor{Mycolor}{\textbf{EXAMPLE~\theexample. #1}} \rmfamily}{\medskip}

\newmdenv[linecolor=black,skipabove=\topsep,skipbelow=\topsep,
leftmargin=-5pt,rightmargin=-5pt,
innerleftmargin=5pt,innerrightmargin=5pt]{textbox}

\definecolor{dkgreen}{rgb}{0,0.6,0}
\definecolor{gray}{rgb}{0.5,0.5,0.5}
\definecolor{mauve}{rgb}{0.58,0,0.82}
\lstset{
%   frame=tb,
  language=Java,
  aboveskip=3mm,
  belowskip=3mm,
  showstringspaces=false,
  columns=flexible,
  basicstyle={\small\ttfamily},
  numbers=none,
  numberstyle=\tiny\color{gray},
  keywordstyle=\color{blue},
  commentstyle=\color{dkgreen},
  stringstyle=\color{mauve},
  breaklines=true,
  breakatwhitespace=true,
  tabsize=3
}



\title{Small dense subgraphs of polarity graphs and the extremal number for the 4-cycle}
\author{
    Michael Tait\thanks{Department of Mathematics, University of California San Diego, \url{mtait@math.ucsd.edu}}
    \and
    Craig Timmons\thanks{Department of Mathematics and Statistics, California State University Sacramento, \mbox{\url{craig.timmons@csus.edu}}}
}

\begin{document}
\maketitle

\begin{abstract}
    In this note, we show that for any $m \in \{1,2, \dots , q +1 \}$, if $G$ is a polarity graph of a projective plane of order $q$ that has an oval, then $G$ contains a subgraph on $m + \binom{m}{2}$ vertices with $m^2+\frac{m^4}{8q} - O ( \frac{m^4}{q^{3/2} } +m )$ edges. As an application, we give the best known lower bounds on the Tur\'{a}n number $\mathrm{ex}(n, C_4)$ for certain values of $n$. In particular, we disprove a conjecture of Abreu, Balbuena, and Labbate concerning $\mathrm{ex}(q^2-q-2, C_4)$ where $q$ is a power of $2$.
\end{abstract}

\section{Introduction}

Let $F$ be a graph. A graph $G$ is said to be $F-free$ if $G$ does not contain $F$ as a subgraph.

Let ex$(n, F)$ denote the $Turán number$ of $F$, which is the maximum number of edges
in an $n$-vertex $F-free$ graph. Write Ex$(n, F )$ for the family of $n$-vertex graphs that
are $F-free$ and have ex$(n, F)$ edges. Graphs in the family Ex$(n, F)$ are called $extremal\
    graphs$. Determining ex$(n, F)$ for different graphs $F$ is one of the most well-studied
problems in extremal graph theory. A case of particular interest is when $F = C_4$,
the cycle on four vertices.




\bibliographystyle{plain}
\bibliography{ref.bib}


\end{document}