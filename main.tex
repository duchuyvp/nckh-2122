\documentclass[12pt]{article}
\usepackage{vntex}
\usepackage[utf8]{inputenc}
\usepackage[T5]{fontenc}
% \usepackage[english, vietnamese]{babel}
\usepackage{tikz}
\usepackage[left=3.00cm, right=2.00cm, top=2.00cm, bottom=2.00cm]{geometry}
\usepackage[unicode]{hyperref}
\usepackage{amsmath}
\usepackage{amssymb}
\usepackage{graphicx}
\usepackage{enumitem}
\usepackage{a4wide,amssymb,epsfig,latexsym,array,hhline,fancyhdr}
\usepackage[normalem]{ulem}
\usepackage[makeroom]{cancel}
\usepackage{amsthm}
\usepackage{multicol,longtable,amscd}
\usepackage{diagbox}
\usepackage{booktabs}
\usepackage{alltt}
\usepackage[framemethod=tikz]{mdframed}
\usepackage{caption,subcaption}
\usepackage{listings}
\usepackage{color}
\usepackage{xcolor}
\usepackage{lipsum}
\usepackage{setspace}
\usepackage{titling}
\usepackage{multicol}
\usepackage{indentfirst}
\usepackage{float}
\usepackage{titlesec}

\usetikzlibrary{decorations}
\usetikzlibrary{decorations.pathreplacing}
\usetikzlibrary{decorations.pathreplacing,calligraphy}
\usetikzlibrary{arrows.meta}
\usetikzlibrary{quotes}
\usetikzlibrary{intersections}
\usetikzlibrary{calc}

\setstretch{1.25}

\newtheorem{theorem}{Định lý}
\newtheorem{corollary}{Hệ quả}
\newtheorem{lemma}{Bổ đề}
\newtheorem*{remark}{Chú ý}
\newtheorem{definition}{Định nghĩa}
\newtheorem*{recap}{Tóm lại}

\newenvironment{solution}
  {\renewcommand\qedsymbol{$\blacktriangleleft $}\begin{proof}[\textcolor{Mycolor}{Giải}]}
  {\end{proof}}

\newcounter{example}
\newenvironment{example}[1][]{\refstepcounter{example}\par\medskip
    \noindent \textcolor{Mycolor}{\textbf{VÍ DỤ~\theexample. #1}} \rmfamily}{\medskip}

\newmdenv[linecolor=black,skipabove=\topsep,skipbelow=\topsep,
leftmargin=-5pt,rightmargin=-5pt,
innerleftmargin=5pt,innerrightmargin=5pt]{textbox}

\definecolor{dkgreen}{rgb}{0,0.6,0}
\definecolor{gray}{rgb}{0.5,0.5,0.5}
\definecolor{mauve}{rgb}{0.58,0,0.82}
\lstset{
%   frame=tb,
  language=Java,
  aboveskip=3mm,
  belowskip=3mm,
  showstringspaces=false,
  columns=flexible,
  basicstyle={\small\ttfamily},
  numbers=none,
  numberstyle=\tiny\color{gray},
  keywordstyle=\color{blue},
  commentstyle=\color{dkgreen},
  stringstyle=\color{mauve},
  breaklines=true,
  breakatwhitespace=true,
  tabsize=3
}


\title{Test}
\author{
    Nguyễn Đức Huy \thanks{K64 Máy tính và Khoa học Thông tin}\\
    Hà Nội \\
    Đại học Khoa học Tự Nhiên
}

\begin{document}
% \maketitle

Nguyễn Đức Huy - 19000350

Ví dụ ma trận 3x3

$$ A =  \begin{pmatrix}
        1 & 2 & 3 \\
        5 & 1 & 3 \\
        1 & 6 & 1
    \end{pmatrix}  $$

Ví dụ phép cộng 2 ma trận 3x3

$$ \begin{pmatrix}
        1 & 2 & 3 \\
        5 & 1 & 3 \\
        1 & 6 & 1
    \end{pmatrix} +  \begin{pmatrix}
        4 & 1 & 3 \\
        5 & 2 & 1 \\
        2 & 3 & 6
    \end{pmatrix} =  \begin{pmatrix}
        5  & 3 & 6 \\
        10 & 3 & 4 \\
        3  & 9 & 7
    \end{pmatrix} $$

Ví dụ ma trận chuyển vị

$$ \begin{pmatrix}
        5  & 3 & 6 \\
        10 & 3 & 4 \\
        3  & 9 & 7
    \end{pmatrix}^T =  \begin{pmatrix}
        5 & 10 & 3 \\
        3 & 3  & 9 \\
        6 & 4  & 7
    \end{pmatrix} $$

Ví dụ ma trận đối xứng

$$ \begin{pmatrix}
        5  & 10 & 3 \\
        10 & 3  & 9 \\
        3  & 9  & 7
    \end{pmatrix} $$

Ví dụ định thức ma trận

$$ \begin{vmatrix}
        5 & 10 & 3 \\
        3 & 3  & 9 \\
        6 & 4  & 7
    \end{vmatrix}  = 237 $$

Ví dụ ma trận nghịch đảo

$$ \begin{pmatrix}
        5 & 10 & 3 \\
        3 & 3  & 9 \\
        6 & 4  & 7
    \end{pmatrix}^{-1} = \frac{1}{237} \begin{pmatrix}
        -15 & -58 & 81  \\
        33  & 17  & -36 \\
        -6  & 40  & -15
    \end{pmatrix} $$

\newpage

Cách tìm giá trị riêng và vector riêng

$$ A = \begin{pmatrix}
        -2 & -4 & 2 \\
        -2 & 1  & 2 \\
        4  & 2  & 5
    \end{pmatrix} $$

\begin{enumerate}
    \item Từ định nghĩa mọi vector riêng $\nu$ tương ứng với giá trị riêng $\lambda$ ta có
          $$A\nu = \lambda \nu$$
          Khi đó
          $$ A\nu - \lambda \nu = (A-\lambda I )\nu = 0 $$
          Phương trình có nghiệm khác không khi và chỉ khi:
          $$ \begin{vmatrix} A - \lambda I \end{vmatrix} = 0$$

          $$ A = \begin{vmatrix}
                  -2 -\lambda & -4         & 2          \\
                  -2          & 1 -\lambda & 2          \\
                  4           & 2          & 5 -\lambda
              \end{vmatrix} = -(\lambda+5)(\lambda -3)(\lambda-6) = 0 $$

          \begin{enumerate}
              \item $\lambda = -5$
              \item $\lambda = 3$
              \item $\lambda = 6$
          \end{enumerate}
    \item với mỗi $\lambda$ ta tìm vector riêng tương ứng
          \begin{enumerate}
              \item $\lambda_1 = -5$

                    $A-\lambda_1 I = \begin{pmatrix}
                            3  & -4 & 2  \\
                            -2 & 6  & 2  \\
                            4  & 2  & 11
                        \end{pmatrix} $

                    $A\nu = \lambda \nu$

                    $(A-\lambda I )\nu = 0$

                    Nghiệm tổng quát $X= \begin{pmatrix} -2x_1 \\ -x_1 \\ x_1 \end{pmatrix}$

                    Cho $x_1 = 1, \nu_1 = \begin{pmatrix} -2 \\ -1 \\ 1 \end{pmatrix}$

              \item $\lambda_2 = 3$

                    $A-\lambda_2 I = \begin{pmatrix}
                            -5 & -4 & 2 \\
                            -2 & -2 & 2 \\
                            4  & 2  & 2
                        \end{pmatrix} $

                    $A\nu = \lambda \nu$

                    $(A-\lambda I )\nu = 0$

                    Nghiệm tổng quát $X= \begin{pmatrix} -2x_2 \\ 3x_2 \\ x_2 \end{pmatrix}$

                    Cho $x_2 = 1, \nu_2 = \begin{pmatrix} -2 \\ 3 \\ 1 \end{pmatrix}$

              \item $\lambda_3 = 6$

                    $A-\lambda_3 I = \begin{pmatrix}
                            -8 & -4 & 2  \\
                            -2 & -5 & 2  \\
                            4  & 2  & -1
                        \end{pmatrix} $

                    $A\nu = \lambda \nu$

                    $(A-\lambda I )\nu = 0$

                    Nghiệm tổng quát $X= \begin{pmatrix} \frac{1}{16} x_3 \\ \frac{3}{8}x_3 \\ x_3 \end{pmatrix}$

                    Cho $x_3 = 16, \nu_3 = \begin{pmatrix} 1 \\ 6 \\ 16 \end{pmatrix}$
          \end{enumerate}
\end{enumerate}
\end{document}